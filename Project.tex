%%%%%%%%%%%%%%%%%%%%%%%%%%%%%%%%%%%%%%%%%
% Short Sectioned Assignment
% LaTeX Template
% Version 1.0 (5/5/12)
%
% This template has been downloaded from:
% http://www.LaTeXTemplates.com
%
% Original author:
% Frits Wenneker (http://www.howtotex.com)
%
% License:
% CC BY-NC-SA 3.0 (http://creativecommons.org/licenses/by-nc-sa/3.0/)
%
%%%%%%%%%%%%%%%%%%%%%%%%%%%%%%%%%%%%%%%%%

%----------------------------------------------------------------------------------------
%	PACKAGES AND OTHER DOCUMENT CONFIGURATIONS
%----------------------------------------------------------------------------------------

\documentclass[paper=letter, fontsize=11pt]{scrartcl} % A4 paper and 11pt font size

\usepackage[T1]{fontenc} % Use 8-bit encoding that has 256 glyphs
%\usepackage{fourier} % Use the Adobe Utopia font for the document - comment this line to return to the LaTeX default
\usepackage[english]{babel} % English language/hyphenation
\usepackage{amsmath,amsfonts,amsthm} % Math packages

\usepackage{sectsty} % Allows customizing section commands

%these two packages are needed for xfig 
\usepackage[dvips,final]{graphicx}
\usepackage{color} %include even if images aren’t in color




\allsectionsfont{\centering \normalfont\scshape} % Make all sections centered, the default font and small caps

\usepackage{fancyhdr} % Custom headers and footers
\pagestyle{fancyplain} % Makes all pages in the document conform to the custom headers and footers
\fancyhead{} % No page header - if you want one, create it in the same way as the footers below
\fancyfoot[L]{} % Empty left footer
\fancyfoot[C]{\thepage} % Empty center footer
\fancyfoot[R]{} % Page numbering for right footer
\renewcommand{\headrulewidth}{0pt} % Remove header underlines
\renewcommand{\footrulewidth}{0pt} % Remove footer underlines
\setlength{\headheight}{13.6pt} % Customize the height of the header

%\numberwithin{equation}{section} % Number equations within sections (i.e. 1.1, 1.2, 2.1, 2.2 instead of 1, 2, 3, 4)
%\numberwithin{figure}{section} % Number figures within sections (i.e. 1.1, 1.2, 2.1, 2.2 instead of 1, 2, 3, 4)
%\numberwithin{table}{section} % Number tables within sections (i.e. 1.1, 1.2, 2.1, 2.2 instead of 1, 2, 3, 4)

\setlength\parindent{0pt} % Removes all indentation from paragraphs - comment this line for an assignment with lots of text

%----------------------------------------------------------------------------------------
%	TITLE SECTION
%----------------------------------------------------------------------------------------

\newcommand{\horrule}[1]{\rule{\linewidth}{#1}} % Create horizontal rule command with 1 argument of height

\title{	
\normalfont \normalsize 
\textsc{University of Lethbridge} \\ [25pt] % Your university, school and/or department name(s)
\horrule{0.5pt} \\[0.4cm] % Thin top horizontal rule
\huge Computational Optimization Term Project Report\\ % The assignment title
\horrule{2pt} \\[0.5cm] % Thick bottom horizontal rule
}

\author{Ahamad Imtiaz Khan\\ID:001188188} % Your name

\date{\normalsize\today} % Today's date or a custom date

\begin{document}

\maketitle % Print the title

%----------------------------------------------------------------------------------------
%	PROBLEM 1
%----------------------------------------------------------------------------------------
\begin{center}
\LARGE Problem 1
\end{center}

\Large \textbf{1. Problem Statement:}
\normalsize Sarah is an Operation Management Student of Takshashila University (TU). In her 
last semester she will need to take some courses.Some courses are compulsory as some courses have choices.
For choosing the courses Sarah need to fulfill some conditions.They are stated as bellow:
    
\begin{enumerate}
\item She has to take 5 courses in total.
\item She has to take Business Strategy (MGT 490) and International Finance (FIN 358).
\item She has interest in service-learning course and found Intergenerational Computing (CIS
102T) and Web Design for Nonprofit Organizations (CIS 102W) interesting. She will take one of the two courses 
between these.
\item Between four potential finance elective courses (Data Analysis in Finance (FIN 325),
 Risk Management(FIN 352), Options, Futures, and Swaps (FIN 356) and 
Fixed Instruments and Markets (FIN 359)) she will take two courses.
\end{enumerate}

Sarah rated each section of the available courses based on content of the course, repuation of the instructor and timing of
the course. Now it's my job to find an appropriate schedule for Sarah which will meet all the above mentioned conditions
and also will give highest total rating. I will solve the problem using two strategies. First one is Greedy Heuristic and 
the second one is LP model. Using Greedy Heuristic I will find a solution which may not be optimal but good enough, using
LP model I will find optimal solution and finally I will find maximum rating of three days a week schedule by extending 
the LP   \\

\Large \textbf{2. Input Data:}
\normalsize The input data for the problem is rearranged as follows: 

\begin{table}[h]
\begin{center}
    \resizebox{\textwidth}{!}{\begin{tabular}{| l | l | l | l | l | l | l | l | l |}
  
    \hline
    Meeting Time(s) & MGT 490 & FIN 358 & CIS 102T & CIS 102W & FIN 325 & FIN 352 & FIN 356 & FIN 359 \\ \hline
    M 6-8:45 p.m. & 4.3 & 0 & 0 & 0 & 0 & 0 & 0 & 0 \\ \hline
    T 6-8:45 p.m. & 0 & 0 & 0 & 0 & 0 & 0 & 0 & 0 \\ \hline
    W 6-8:45 p.m. & 0 & 0 & 0 & 0 & 0 & 0 & 0 & 0 \\ \hline
    F 6-8:45 p.m. & 0 & 0 & 0 & 0 & 0 & 0 & 0 & 0 \\ \hline
    \shortstack[l]{M 1:25-2:20 p.m. \\ W 1:25-3:15 p.m.} & 0 & 0 & 0 & 0 & 0 & 0 & 0 & 0 \\ \hline
    \shortstack[l]{T 1:25-3:15 p.m. \\ Th 1:25-2:20 p.m.} & 0 & 0 & 0 & 0 & 0 & 0 & 0 & 0 \\ \hline
    W 2:30-5:15 p.m. & 0 & 0 & 0 & 0 & 0 & 0 & 0 & 0 \\ \hline
    Th 2:30-5:15 p.m. & 0 & 0 & 0 & 0 & 0 & 0 & 0 & 0 \\ \hline
    Th 6-8:45 p.m. & 0 & 0 & 0 & 0 & 0 & 0 & 0 & 0 \\ \hline
    \shortstack[l]{M 1:25-3:15 p.m. \\ W 1:25-2:20 p.m.} & 0 & 0 & 0 & 0 & 0 & 0 & 0 & 0 \\ \hline
    
    \end{tabular}}
    \end{center}

\caption{Course Data}

\end{table}

Table 1 shows the rating of the sections of corresponding subjects and their Meeting times. Ratings of sections of a subject 
are in the same column under the subject code and the corresponding rows indicates the meeting times of the sections. For
 example MGT 490 has 6 sections and the first section has rating of 4.3 and its meeting time is M 6-8:45 p.m.    



\Large \textbf{3. Solution Strategies:}
\normalsize \textbf{a.Heuristic} 




%----------------------------------------------------------------------------------------

\end{document}